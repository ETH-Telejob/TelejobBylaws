
\documentclass[10pt]{article}
\usepackage{graphicx}  
\usepackage{multirow}
\usepackage{calc}
\usepackage[german]{babel} 
\reversemarginpar
\usepackage[paper=letterpaper,
            %includefoot, % Uncomment to put page number above margin
            marginparwidth=1.2in,     % Length of section titles
            marginparsep=.05in,       % Space between titles and text
            margin=0.8in,               % 1 inch margins
            includemp]{geometry}

\setlength{\parindent}{0in}

\usepackage[shortlabels]{enumitem}

\newcounter{qcounter}

\makeatletter
\newlength{\bibhang}
\setlength{\bibhang}{1em}
\newlength{\bibsep}
 {\@listi \global\bibsep\itemsep \global\advance\bibsep by\parsep}
\newlist{bibsection}{itemize}{3}
\setlist[bibsection]{label=,leftmargin=\bibhang,%
        itemindent=-\bibhang,
        itemsep=\bibsep,parsep=\z@,partopsep=0pt,
        topsep=0pt}
\newlist{bibenum}{enumerate}{3}
\setlist[bibenum]{label=[\arabic*],resume,leftmargin={\bibhang+\widthof{[999]}},%
        itemindent=-\bibhang,
        itemsep=\bibsep,parsep=\z@,partopsep=0pt,
        topsep=0pt}
\let\oldendbibenum\endbibenum
\def\endbibenum{\oldendbibenum\vspace{-.6\baselineskip}}
\let\oldendbibsection\endbibsection
\def\endbibsection{\oldendbibsection\vspace{-.6\baselineskip}}
\makeatother


\usepackage{fancyhdr,lastpage}
\pagestyle{fancy}
%\pagestyle{empty}      % Uncomment this to get rid of page numbers
\fancyhf{}\renewcommand{\headrulewidth}{0pt}
\fancyfootoffset{\marginparsep+\marginparwidth}
\newlength{\footpageshift}
\setlength{\footpageshift}
          {0.5\textwidth+0.5\marginparsep+0.5\marginparwidth-2in}
\lfoot{\hspace{\footpageshift}%
       \parbox{4in}{\, \hfill %
                    \arabic{page} of \protect\pageref*{LastPage} % +LP
%                    \arabic{page}                               % -LP
                    \hfill \,}}

\usepackage{color,hyperref}
\definecolor{darkblue}{rgb}{0.0,0.0,0.3}
\hypersetup{colorlinks,breaklinks,
            linkcolor=darkblue,urlcolor=darkblue,
            anchorcolor=darkblue,citecolor=darkblue}

\newcommand{\makeheading}[2][]%
        {\hspace*{-\marginparsep minus \marginparwidth}%
         \begin{minipage}[t]{\textwidth+\marginparwidth+\marginparsep}%
             {\large \bfseries #2 \hfill #1}\\[-0.15\baselineskip]%
                 \rule{\columnwidth}{1pt}%
         \end{minipage}}

\renewcommand{\section}[1]{\pagebreak[3]%
    \vspace{1.3\baselineskip}%
    \phantomsection\addcontentsline{toc}{section}{#1}%
    \noindent\llap{\scshape\smash{\parbox[t]{\marginparwidth}{\hyphenpenalty=10000\raggedright #1}}}%
    \vspace{-\baselineskip}\par}



\newcommand{\blankline}{\quad\pagebreak[3]}
\newcommand{\halfblankline}{\quad\vspace{-0.5\baselineskip}\pagebreak[3]}


\newcommand\doilink[1]{\href{http://dx.doi.org/#1}{#1}}
\newcommand\doi[1]{doi:\doilink{#1}}


\providecommand*\url[1]{\href{#1}{#1}}

\renewcommand*\url[1]{\href{#1}{\texttt{#1}}}


\providecommand*\email[1]{\href{mailto:#1}{#1}}

\providecommand\BibTeX{{B\kern-.05em{\sc i\kern-.025em b}\kern-.08em
    \TeX}}
\providecommand\Matlab{\textsc{Matlab}}
\hyphenation{bio-mim-ic-ry bio-in-spi-ra-tion re-us-a-ble pro-vid-er}

%%%%%%%%%%%%%%%%%%%%%%%% End Helper Commands %%%%%%%%%%%%%%%%%%%%%%%%%%%

%%%%%%%%%%%%%%%%%%%%%%%%% Begin Bylaws Document %%%%%%%%%%%%%%%%%%%%%%%%%%%%

\begin{document}
\makeheading{Telejob Bylaws\\
\small{Approved by the Telejob Executive Meeting on XXXXXXX}
}



\vspace{30pt}
\begin{list}{{\bf Article \arabic{qcounter}:~}}{\usecounter{qcounter}}

\item[] \hspace{-30pt}\section {Fundamentals}
\vspace {-23pt}

\item {\bf Legal Definition}\\
Telejob (the Organization) is a voluntary based non-profit organization operated by the Academic Association of Scientific Staff at ETH Zurich (AVETH).

\item {\bf Location and Duration}\\
The Organization's headquarters are located in Sonneggstrasse 33, 8092 Zurich, Switzerland. The Organization was founded in 1989 and shall be of unlimited duration.

\item {\bf Purpose}\\
$^{1}$The Organization serves the purpose of supporting ETH's academic community members in their search for employment. This includes all current and former students and scientific staff from the ETH domain.\\ 
$^{2}$The Organization fosters the culture of entrepreneurship and voluntary business activity within ETH's academic community.\\
$^{3}$The Organization seeks to attract and develop members interested in voluntary business activities by providing a corporate platform focusing on utisiling novel technologies and member education.

\item {\bf Resources}\\
The Organization's resources are derived from revenues of business activities, sponsorship, donations and legacies and public subsidies.

\item[] \hspace{-30pt}\section {Membership}
\vspace {-23pt}

\item {\bf Prerequisite}\label{prerequisite}\\
Any person may become a member of the Organization after demonstrating dedication to the goals of the Organization through their commitments or actions. It is preferred (but not required) that the person was or currently is a student of a higher-education institution in Switzerland.

\item {\bf Obtaining Membership}\\
To become a member of the Organization, one must first fulfill the prerequisite in \hyperref[prerequisite]{Article~\ref{prerequisite}} and declare a written intention of becoming an Organization member. The membership is intact after the approval of the Executive Meeting (\hyperref[EM]{Article~\ref{EM}}).

\item {\bf Cessation of Membership}\\ \label{cessation}
The membership ceases\\ 
$^{1}$by a written resignation to the Executive Board (\hyperref[EB]{Article~\ref{EB}}), effective immediately;\\
$^{2}$by an exclusion order from the Executive Board at least 14 days before the next Executive Meeting, in which the cessation can only become intact through an EM decision;\\
$^{3}$on death;

\item {\bf Rights and Duties of Members}\label{Duties}\\
$^{1}$Every member has the right to vote in EM elections and petitions\\
$^{2}$Every member has the right to submit petitions at the EM. Petitions to change the Bylaws have to be submitted at least 7 days prior and only to the first EM after an AVETH General Assembly (\hyperref[GA]{Article~\ref{GA}}).\\
$^{4}$Members have the right to call for an extraordinary EM. This EM has to be organized by the Executive Board within 14 days.\\

\item {\bf The imperative of volunteering work}\\
All members of the Organization work on a volunteer basis and as such can only be reimbursed for their actual expenses and travel costs. Potential attendance fees cannot exceed those paid for official commissions. For activities beyond the usual function, each Committee member is eligible for appropriate compensation, approved by the Presidency and the Treasurer. \\
$^{1}$Paid employees are not eligible for a position in the Executive Board.\\
$^{2}$Paid employees are automaticaly members of the Organization.

\item[] \hspace{-30pt}\section {Organization}
\vspace {-23pt}

\item {\bf List of Organs}\\\label{ListOfOrgans}
The Organization consists of the following organs:\\
$^{1}$ The AVETH General Assembly (GA) and the AVETH Board\\
$^{2}$ The Executive Meeting (EM)\\
$^{3}$ The Executive Board (EB)\\
$^{4}$ The Advisory Board (AB)\\
$^{5}$ The Corporate Revision (R)

\item {\bf Fiscal Year}\\
The Organization's fiscal year coincides with the calendary year, i.e. January 1$^{\mathrm{st}}$ to December 31$^{\mathrm{st}}$.

\item[] \hspace{-30pt}\section {Relation to AVETH}
\vspace {-23pt}

\item {\bf AVETH General Assembly (GA)}\label{GA}\\
The GA is the Organization's supreme authority. It is composed of all AVETH members and takes place at least once per fiscal year. The GA elects the Organization's Presidency, Treasury and other members of the Executive Board. Once per year, the Organization reports its activities and financial results, nominates candidates of its Executive Board and files petitions.
 
\item {\bf AVETH Board}\\
The Organizations Executive Board must periodically inform the AVETH Board of the business and its operation. The AVETH board has right to inspection of the finances of the Organization, at any time.

\item[] \hspace{-30pt}\section {Executive Meeting (EM)}
\vspace {-23pt}

\item {\bf Competences of the Executive Meeting}\label{EM}\\
The Executive Meeting (EM) is the highest body in forming official decisions of the Organization. It is composed of all members as well as all organs of the Organization. Typical competences of the Executive Meeting are\\
$^{1}$appointments and dismissals of members of the Organization;\\
$^{2}$supervision of activities within the Organization;\\
$^{3}$amendment of the Bylaws;\\
$^{4}$decision on official reports, candidate nominations and petitions the Organization presents at the GA.\\

\item {\bf Basics}\\
$^{1}$The Organization holds at least one EM each month.\\
$^{2}$The EM has to be announced at least 7 days in advance towards at least all organs of the Organization (\hyperref[ListOfOrgans]{Article~\ref{ListOfOrgans}}). Amendment proposals of the Bylaws must be inlcuded in this announcement can only be addressed in the first EM after an AVETH GA.\\
$^{3}$The EM is chaired by the Presidency.\\
$^{4}$Upon procedural request, the chair of the EM for individual agenda points or for the whole EM may be transferred to a differnet person.\\
$^{5}$Individuals outside the Organizations organs can only attend the EM after the EM's approval. 

\item {\bf Validity}\\
The EM shall be considered valid with\\
$^{1}$at least one member of the Presidency present; \\
$^{2}$at least half (rounded up) of the Executive Board members present; \\
$^{3}$a meeting protocol approved by the Executive Meeting; \\

\item {\bf Decision Procedures}\label{EMDecisionProcedure}\\
$^{1}$Decision of the EM shall be taken by a simple majority vote of the members present. In case of a deadlock, the Presidency shall have the deciding vote.\\
$^{2}$Decision concerning the amendment of the Bylaws must be approved by a two-thirds majority of the members present.\\
$^{3}$The Advisory Board (AB, \hyperref[AB]{Article~\ref{AB}}) has a veto vote on all decisions made by the EM. This veto vote is decided through a simple majority vote among the AB members present. \\
$^{4}$Votes are done by a show of hands. Voting can also take place by secret ballot, if at least three members request it. \\

\item {\bf Mandatory Agenda Points}\\
The agenda of the EM must include at least\\\\
$^{1}$Approval of the meeting protocol of the previous EM;\\
$^{2}$An agenda proposal presented by the EM chair, which is to be approved by the EM;\\
$^{3}$A report of the Treasurery on current financial status; In cases of absence, the Treasury can be substituted by the Presidency; \\
$^{4}$Report of the EB on monthly activities;\\
In addition to the agenda points mentioned above, all issues and discussions can be raised through a request from a member present. The chair then proposes the adapted EM agenda, which is to be approved by the EM;\\

\item[] \hspace{-30pt}\section {Executive Board (EB)}
\vspace {-23pt}

\item {\bf Purpose and Competence}\label{EB}\\
The Executive Board (EB) is authorized by the GA to carry out all acts that furthe the purposes of the Organization. It has the most extensive power to manage the Association's day-to-day affairs. The EB decides with a simple majority of its members at a meeting. 

\item {\bf Decision Competence between EMs}\\
The Executive Board can make urgent decisions on behalf of the Organization between EMs. Any EB member can launch an urgent decision process by a written proposal to all other EB members, who can then participate with their votes until 24 hours after the proposal message. Delayed messages are counted as abstentions. The Presidency as well as the Treasury have the right to postpone the proposal to the next EM. The decisions conducted by this article must be reported at the next EM.


\item {\bf Executive Board members Members}\\
The Executive Board is composed of members of the Organization elected by the GA under the Organization's recommendation. The official recommendation of the Organization is elected by the Executive Meeting in the following competencies:\\ \\
$^{1}$ Presidency (two co-Presidents or a President and a Vice-President)\\
$^{2}$ Treasurer\\
$^{3}$ Vice Presidency of Customer Relations (VP-CR)\\
$^{4}$ Vice Presidency of Student Relations (VP-SR)\\
$^{5}$ Vice Presidency of Technology (VP-T)\\
$^{6}$ Vice Presidency of Team Development (VP-TD)\\
$^{7}$ Vice Presidency of Strategy (VP-S)\\
$^{8}$ Director of PolyHACK (PHD)\\
$^{9}$ Director of ETH Gethired (D-ETHGH)\\
$^{10}$ Director of Polyclub (D-PC)\\\\
At minimum, three members covering the seats in Presidency (two members) and Treasury (one member) need to be elected. 

\item {\bf The Presidency and the Treasurer}\label{P&T}\\
The members of the Presidency and the Treasurer are automatically members of the AVETH board, where they represent Telejob. The AVETH GA elects both the Presidency members and the Treasurer. Together, they are responsible for the activities of the Organization. \\\\
$^{1}$The Presidency and the Treasurer together can sign legal transactions concerning the Organization in the name of AVETH.\\
$^{2}$The Presidency leads the EM and represents the Organization in all cases where no other, clearly defined, representations are given in the Bylaws or by the EB. \\
$^{3}$The Presidency develops and executes the Organization's strategy with the goal to fulfill it's purpose sustainably.\\
$^{3}$The Treasurer administers the finances and informs the Organization on the financial statement as well as the budget.\\
$^{4}$The terms of the Presidency as well as the Treasurer shall last for one year and is renewable.

\item {\bf Suspension of Presidency and Treasurer}\\
Only if the Presidency or the Treasurer do not fulfill the responsibilities mentioned in \hyperref[P&T]{Article~\ref{P&T}}, the EB can order an exclusion by the procedure following \hyperref[cessation]{Article~\ref{cessation}} with the following exceptions:\\
$^{1}$The targeted cantidate remains the current position until the GA, where he/she will not be nominated as cadidate by the Organization;\\
$^{2}$For an earlier suspension the EM can order an extraordinary GA according to Article 10 Paragraph 4 of the AVETH Bylaws;

\item {\bf Indisposition of the Presidency}\\
In the case of the indisposition of all members in the Presidency, a supermajority (at least 60\%) of the EB will elect a President in a provisional capacity with their mandate lasting until the next EM. 

\item {\bf Vice Presidents and Directors}\\\label{VP&D}
Vice Presidents and Directors are responsible for the sustainable success of a defined ressort within the Organization. The responsibility include:\\
$^{1}$Financial competence for the respective ressort, including budget proposal and financial statement;\\ 
$^{2}$Active participation of at least 10 EMs per calendar year;\\
$^{3}$Leading team members from the respective ressort;\\

\item {\bf Suspension of Vice Presidents and Directors}\\
Only if the Vice President of the Director does not fulfill the responsibilities mentioned in \hyperref[VP&D]{Article~\ref{VP&D}}, the EB can order an exclusion by the procedure following \hyperref[cessation]{Article~\ref{cessation}}. The candidate then gets suspended of all competences after the respective EM decision and will not be nominated as a candidate at the next GA.

\item[] \hspace{-30pt}\section {Advisory Board (AB)}
\vspace {-23pt}

\item {\bf Purpose of the Advisory Board}\label{AB}\\
The AB serves as a consulting and supervising role towards the Organization. Through the experience of its members, the AB should help the Organization fulfill its purpose sustainably. 

\item {\bf Prerequisite}\label{prerequisite AB}\\
Any person may become a member of the Advisory Board with at least one of the following prerequisites fulfilled:\\
$^{1}$ Experience in the business world in Switzerland;
$^{2}$ Experience in running an association within the ETH ecosystem; 
$^{3}$ Demonstrating dedication to the goals of the Organization through commitments or actions; 
 

To become a member of the Advisory Board, one must first fulfill the prerequite in \hyperref[prerequisite AB]{Article~\ref{prerequisite AB}} and declare an intention of becoming an Organization member. The membership is intact after the approval of the Advisory Board. 

\item {\bf Cessation of Membership}\\
The membership ceases\\
$^{1}$ by a written resignation to the Advisory Board, effective immediately;\\
$^{2}$ by an exclusion order from any AB member, which becomes intact after a supermajority (at least 60\%) of AB members agreeing to it;\\
$^{3}$ on death; 

\item {\bf Competences of the Advisory Board}\\
$^{1}$ Following \hyperref[EMDecisionProcedure]{Article~\ref{EMDecisionProcedure}}, the AB has a veto vote on all decisions at the EM;\\
$^{2}$ The AB can call for an extraordinary EM, which must be organized by the EB lastest 14 days after the call.\\
$^{2}$ AB members can not be members of the Organization. Members of the Organization can not be AB members.

\item {\bf Decision Procedures of the Advisory Board}\\
$^{1}$ Any decisions of the AB shall be taken by a simple majority vote of the members. In case of a deadlock, the Advisory Board does not make the decision. \\
$^{2}$ Any member of the AB can propose a decision at any time, which needs to be informed to all other AB members. The other AB members then have maximum of 336 hours (14 days) to discuss and vote on this decision. Any vote after this time limit is considered as abstaining the vote.  


\item[] \hspace{-30pt}\section {Corporate Revision}
\vspace {-23pt}

\item {\bf Auditors}\\
The finances of the Organization is audited by a professional auditor, who cannot be a member is any other organ of the Organization.


\item[] \hspace{-30pt}\section {Various Provisions}
\vspace {-23pt}

\item {\bf Dissolution}\\
Should the Organization be dissolved, the available assets should be transferred to a non-profit organization pusuing public interest goals similar to those of the Organization. The choice is made by and at the GA that dissolves the Organization. Under no circumstances should the assets be returned to members of any organs of the Organization, nor should they use a part or a total of assets for their own benefit.

\item {\bf Approval and Implementation}\\
These Bylaws were last changed on XXXXXX by the EM of the Organization. they replace the previous version from XXXXX. The take effect immediately.

\end{list}


  
\end{document} 